\documentclass[runningheads]{llncs}

% \usepackage[numbers]{natbib}
% \renewcommand\bibsection{\section*{\refname}\small\renewcommand\bibnumfmt[1]{##1.}}
\usepackage[utf8]{inputenc}
\usepackage{amsmath}
\usepackage{amsfonts,amssymb}
\usepackage{hyperref}
% \usepackage{xspace}
% \usepackage{amsthm}
% \usepackage{enumitem} % topsep

\usepackage{tikz}
% TikZ libraries
\usetikzlibrary{myautomata}
\usetikzlibrary{fit, backgrounds}

\newenvironment{nstabbing}
  {\setlength{\topsep}{0pt}%
   \setlength{\partopsep}{0pt}%
   \tabbing}
  {\endtabbing}

%{{{ (La)TeX definitions

\newcommand{\Nset}{{\mathbb{N}_0}}

\newcommand{\todo}[1]{\textcolor{red}{#1}}
\def\Inf{\ensuremath{\mathsf{Inf}}}
\def\Fin{\ensuremath{\mathsf{Fin}}}
\def\ttrue{\ensuremath{\mathsf{t}}}
\def\ffalse{\ensuremath{\mathsf{f}}}
\def\CQ{\mathcal{C}}
\def\BigO#1{\ensuremath{\mathcal{O}\left(#1\right)}}
\newcommand{\B}{\texorpdfstring{\ensuremath{\mathds{B}}}{B}}
\def\isempty{\textsc{is\_empty}\xspace}
\def\sccsof{\textsc{SCCs\_of}\xspace}
\def\issccempty{\textsc{is\_SCC\_empty}\xspace}
\def\marksof{\textsc{marks\_of}\xspace}
\def\remove{\textsc{remove}\xspace}

\def\Mocc{M_{\mathrm{occur}}}
\def\Mevery{M_{\mathrm{every}}}
\newcommand{\G}{\mathbf{G}}
\newcommand{\CA}{\ensuremath{\mathcal{A}}}
\newcommand{\CB}{\ensuremath{\mathcal{B}}}
\def\init{\iota}
\def\lang{L} % FIX this
\newcommand{\cM}{\mathcal{M}}
\newcommand{\bscc}{\mathcal{B}}
\newcommand{\sched}{\mathfrak{s}}
\newcommand{\Act}{\mathit{Act}}
\newcommand{\NP}{\textsf{NP}}
\newcommand{\POLY}{\textsf{P}}
\newcommand{\class}{\ensuremath{\mathcal{C}}}
\newcommand{\finally}{\mathsf{F}}
\newcommand{\globally}{\mathsf{G}}
% \newcommand{\finally}{\mathbf{F}}
% \newcommand{\globally}{\mathbf{G}}

% Remove space from footnote after interpunction. Usage:
% \punct{.}\footnote{footnotetext}
\newlength{\spc} % declare a variable to save spacing value
\newcommand{\punct}[1]{%
  \settowidth{\spc}{#1}% set value of \spc variable to the width of #1 argument
  \addtolength{\spc}{-1.8\spc}% subtract from \spc about two (1.8) of its values making its magnitude negative
  #1% print the optional argument
  \hspace*{\spc}% print an additional negative spacing stored in \spc after #1
}

%}}}

\begin{document}

%{{{ Title and authors

\title{Reducing Acceptance Condition\\ of Emerson-Lei Automata}
\authorrunning{T.~Schwarzová and J. Strej\v{c}ek}
 \author{Tereza Schwarzová \and Jan Strej\v{c}ek}
 \institute}}

\maketitle

%{{{ Abstract

\begin{abstract}
  We present an QBF-based algorithm that reduces the number of
  acceptance sets of transition-based Emerson-Lei automata.
\end{abstract}

%}}}

\section{Introduction}
\begin{itemize}
\item Emerson-Lei automata jsou v principu staré, ale v poslední době
  znovuobjevene~\cite{babiak.15.cav} a je o tom ted spousta
  clanku~\cite{baier.19.atva} a nastroju
\item algoritmy jsou obvykle citlive na akceptacni podminku (citace)
\item cilem prace je zjednodusit akceptacni podminku (zejmnena
  redukovat pocet akceptacnich znacek) bez zmeny struktury automatu
\end{itemize}

\section{Preliminaries}
Definice TELA, popsat QBF

\section{Reduction based on acceptance sets relations}
Puvodni telatko

\section{QBF-based reduction}
\begin{itemize}
\item obecny popis konstrukce formule
\item (level 3)
\item level 4
\end{itemize}

\section{Implementation and experimental evaluation}

\section{Conclusions}


%\bibliographystyle{abbrvnat}
\bibliographystyle{plain}
\bibliography{mc}


\end{document}


%%% Local Variables:
%%% mode: latex
%%% End:
