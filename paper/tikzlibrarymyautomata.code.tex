% Automata stuff
\usetikzlibrary{automata}
\usetikzlibrary{arrows.meta}
\usetikzlibrary{bending}
\usetikzlibrary{shapes.callouts}
\usetikzlibrary{quotes}
\usetikzlibrary{positioning}
\usetikzlibrary{calc}

\tikzstyle{automaton}=[
  % Use nice arrows that do not touch their destination.
  thick,shorten >=1pt,>={Stealth[round,bend]},
  % Encourage a common distance between all states.
  node distance=1.8cm,
  % Disable the "start" text in front of the initial arrow.
  initial text=,
  % Allows to use scale to scale the figure
  transform shape,
  % Reduce the size of the hidden node at the beginning of the initial arrow.
  every initial by arrow/.style={every node/.style={inner sep=0pt}},
  % Encourage a common size of all states that is smaller than the default.
  every state/.style={text width=8mm,inner sep=0pt,align=center,fill=white}
]
\tikzstyle{smallautomaton}=[
  automaton,
  node distance=5mm,
  every state/.style={minimum size=4mm,fill=white,inner sep=1pt}
]

\tikzstyle{cstate}=[state,capsule,text width=]
\tikzstyle{dot}=[fill=black,circle,minimum size=5pt,inner sep=0]
\tikzstyle{acclabel} = [fill=yellow!20,rounded corners=3pt,draw=black]
\tikzstyle{flabel} = [rounded corners=3pt,draw=black,inner sep=5pt]
\tikzstyle{unreachable} = [densely dotted]

\makeatletter
\tikzoption{initial angle}{\tikzaddafternodepathoption{\def\tikz@initial@angle{#1}}}
\makeatother

% pass "initial overlay" as a parameter to the tikz figure if you want
% to initial arrow to be ignored from the computation of the figure
% box.  This might be needed for better caption centering.
\tikzstyle{initial overlay}=[every initial by arrow/.style={overlay}]

\tikzset{
  collacc0/.style={fill=blue!50!cyan},
  collacc1/.style={fill=magenta},
  collacc2/.style={fill=orange!90!black},
  collacc3/.style={fill=green!70!black},
  collacc4/.style={fill=blue!50!black},
  collacc5/.style={fill=red!70!black},
  collacc6/.style={fill=grey!70!black},
  collacc7/.style={fill=yellow!90!black},
  collacc8/.style={fill=black!60!green},
  collacc9/.style={fill=cyan!60!gray},
}

% Acceptance sets are displayed as small circles with white
% sans-serif number.  The default color is light blue. Can be
% changed by collacc styles.  Use anchor=center to ignore the
% predefined anchor on edges like "loop above" or "loop left"
% or when using path[auto].
\tikzstyle{accset}=[
  circle,inner sep=.9pt,draw=white,
  collacc0, text=white,
  thin,
  anchor=center,
  font=\bfseries\sffamily\scriptsize,
  minimum size={4pt},
]
% Acceptance set as square
\tikzstyle{accsquare}=[accset,rectangle,inner sep=1.9pt,rounded corners=0pt]
\tikzstyle{accdiamond}=[accset,diamond,inner sep=.7pt,rounded corners=0pt]

\tikzset{
  % Easy typesetting of colored accepting marks. Usage:
  % (n1) edge pic[pos=.3]{acc={COLOR_ID}{LABEL}}
  % For COLOR_ID there must be collaccCOLOR_ID defined
  pics/acc/.style 2 args={
    code={ %
      \node[accset,collacc#1]{#2};
      }
  },
  pics/accsq/.style 2 args={
      code={ %
        \node[accsquare,collacc#1]{#2};
        }
    },
  pics/accdm/.style 2 args={
      code={ %
        \node[accdiamond,collacc#1]{#2};
        }
  },
  % Construction for labels of edges that have some acceptance
  % mark. Shifts label further from the edge to avoid ugly clash
  % with the mark. Usage: pic[pos=.3,left]{l=LABEL}
  pics/l/.style={
      code={ %
        \node[outer sep=2pt] {#1};
        }
    }
}

% Macros for colored acceptance marks used in text. Usage:
% \tacc{COLOR_ID}{LABEL}
\def\tacc#1#2{\tikz[smallautomaton,baseline=-3pt] \pic{acc={#1}{#2}};}
\def\taccsq#1#2{\tikz[smallautomaton,baseline=-3pt] \pic{accsq={#1}{#2}};}
\def\taccdm#1#2{\tikz[smallautomaton,baseline=-3pt] \pic{accdm={#1}{#2}};}
